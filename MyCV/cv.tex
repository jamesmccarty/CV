%%%%%%%%%%%%%%%%%%%%%%%%%%%%%%%%%%%%%%%%%
% Resume/CV
% XeLaTeX Template
% Version 1.0 (22/1/2015)
%
% This template has been downloaded from:
% http://www.LaTeXTemplates.com
%
% Original author:
% Howard Wilson (https://github.com/watsonbox/cv_template_2004) with
% extensive modifications by Vel (vel@latextemplates.com)
%
% License:
% CC BY-NC-SA 3.0 (http://creativecommons.org/licenses/by-nc-sa/3.0/)
%
%%%%%%%%%%%%%%%%%%%%%%%%%%%%%%%%%%%%%%%%%

%----------------------------------------------------------------------------------------
%	PACKAGES AND OTHER DOCUMENT CONFIGURATIONS
%----------------------------------------------------------------------------------------

\documentclass[10pt]{article} % Default font size

\input{structure.tex} % Include the file specifying document layout

%----------------------------------------------------------------------------------------

\begin{document}

%----------------------------------------------------------------------------------------
%	NAME AND CONTACT INFORMATION
%----------------------------------------------------------------------------------------

\title{James McCarty} % Print the main header

%------------------------------------------------

\parbox{0.5\textwidth}{ % First block
\begin{tabbing} % Enables tabbing
\hspace{3cm} \= \hspace{4cm} \= \kill % Spacing within the block
{\bf Assistant Professor} \\
Department of Chemistry and Biochemistry\\ % Address line 1
Western Washington University \\
516 High Street \\ % Address line 2
Bellingham, WA 98225 \\
%{\bf Date of Birth} \> 7$^{th}$ September 1979 \\ % Date of birth 
%{\bf Nationality} \> British % Nationality
\end{tabbing}}
\hfill % Horizontal space between the two blocks
\parbox{0.5\textwidth}{ % Second block
\begin{tabbing} % Enables tabbing
\hspace{3cm} \= \hspace{4cm} \= \kill % Spacing within the block
%{\bf Home Phone} \> +0 (000) 111 1111 \\ % Home phone
{\bf Phone} \> +1 (805) 568 8223 \\ % Mobile phone
{\bf Email} \> \href{mailto:Jay.McCarty@wwu.edu}{Jay.McCarty@wwu.edu} \\ % Email address
{\bf Website} \> \href{https://websitehere}{websitehere} \\
\end{tabbing}}

%----------------------------------------------------------------------------------------
%	EMPLOYMENT HISTORY SECTION
%----------------------------------------------------------------------------------------

\section{Professional Appointments}

\job
{Aug 2019 -}{Present}
{Assistant Professor of Computational Biochemistry}
{Department of Chemistry and Biochemistry}
{Western Washington University, Bellingham, WA, U.S.A.}

%------------------------------------------------

\jobb
{March 2017 -}{July 2019}
{Postdoctoral Fellow}
{Department of Chemistry and Biochemistry }
{University of California Santa Barbara, Santa Barbara, CA, U.S.A.}
{Advisors: Profs. Joan-Emma Shea and Glenn H. Fredrickson} 

\jobb
{Feb 2014 -}{Feb 2017}
{Postdoctoral Fellow}
{Department of Chemistry and Applied Biosciences}
{ETH Z{\"u}rich, Switzerland}
{Advisor: Prof. Michele Parrinello} 

%----------------------------------------------------------------------------------------
%	EDUCATION SECTION
%----------------------------------------------------------------------------------------

\section{Education}
\tabbedblock{
{\bf Ph.D., Chemistry}, December 2013  \\
University of Oregon, Eugene, OR\\
Advisor: Prof. Marina Guenza\\
Doctoral Thesis: ``{\it Multiscale modeling and thermodynamic consistency between soft-particle representations of} \\ {\it macromolecular liquids}"
}

\tabbedblock{
{\bf B.S., Biochemistry}, March 2007 \\
California Polytechnic State University, San Luis Obispo, CA \\
{\it Magna cum laude}\\
}
%----------------------------------------------------------------------------------------
%	PERSONAL PROFILE
%----------------------------------------------------------------------------------------

\section{Research Interests}

Molecular dynamics simulation in life and materials science \\
Statistical mechanics and enhanced sampling of rare events \\
Protein dynamics, biophysics, soft condensed matter \\
Computational simulations of polymer field theories \\
Coarse-graining and liquid state theory\\


\section{Publications}
%Polarizable paper \\
%\\
%Review article \\
%\\
%J. McCarty, K. Delaney, G. H. Fredrickson, and J.-E. Shea, ``experimental design paper for IDPs"\\
%\\
%N. J. Sherck, J. McCarty, K. T. Delaney, J.-E. Shea, M. S. Shell, and G. H. Fredrickson, ``Efficient Coarse-Grained Simulations of Polymers: Molecular Dynamics and Field-Theoretic Simulations," {\it Manuscript in preparation}\\
%\\
%A. Arsiccio, J. McCarty, R. Pisano, and J.-E. Shea, ``The ice-water surface destabilizes the protein structure by enhancing cold denaturation phenomena," {\it Manuscript in preparation} \\
%\\
S. P. O. Danielsen, J. McCarty, J.-E. Shea, K. T. Delaney, and G. H. Fredrickson, ``Small ion effects on self-coacervation phenomena in block polyampholytes,"  {\it J. Chem. Phys.}, 151, 000000 (2019); doi: 10.1063/1.5109045 \\
\\
S. P. O. Danielsen, J. McCarty, J.-E. Shea, K. T. Delaney, and G. H. Fredrickson, ``Molecular design of self-coacervation phenomena in block polyampholytes," {\it Proc. Natl. Acad. Sci. USA}, 116, 8224-8232 (2019) \\
\\
J. McCarty, K. T. Delaney, S. P. O. Danielsen, G. H. Fredrickson, and J.-E. Shea, ``Complete phase diagram for liquid-liquid phase separation of intrinsically disordered proteins" {\it J. Phys. Chem. Lett.}, 10, 1644-1652  (2019)\\
\\
Y. Lin, J. McCarty, J. N. Rauch, K. T. Delaney, K. S. Kosik, G. H. Fredrickson, J.-E. Shea, and S. Han, ``Phase diagram reveals a narrow equilibrium window for the complex coacervation of tau and RNA under cellular conditions, " {\it eLife}, 8:e42571 (2019) [co-first author]\\
\\
A. Arsiccio, J. McCarty, R. Pisano, and J.-E. Shea, ``The effect of surfactants on surface-induced denaturation of proteins: evidence of an orientation-dependent mechanism," {\it J. Phys. Chem. B}, 122, 11390 (2018)\\
\\
M. G. Guenza, M. Dinpajooh, J. McCarty, and I. Y. Lyubimov, ``On the accuracy, transferability, and efficiency of coarse-grained models of molecular liquids," {\it J. Phys. Chem. B}, 122, 10257 (2018) [selected for ACS Editors' Choice]  \\
\\
D. Mendels, J. McCarty, P. Piaggi, and M. Parrinello, ``Searching for entropically stabilized phases: the case of silver iodide," {\it J. Phys. Chem. C}, 122, 1786 (2018) \\
\\
J. McCarty and M. Parrinello, ``A variational conformational dynamics approach to the selection of collective variables in metadynamics," {\it J. Chem. Phys.}, 147, 204109 (2017). \\
\\
G. Piccini, J. McCarty, O. Valsson, and M. Parrinello, ``Variational Flooding Study of a $S_N2$ Reaction," {\it J. Phys. Chem. Lett.},  8, 580 (2017).\\
\\
J. McCarty, O. Valsson, and M. Parrinello, ``Bespoke bias for obtaining free energy differences within variationally enhanced sampling," {\it J. Chem. Theory Comput.}, 12, 2162 (2016). \\
\\
J. McCarty, O. Valsson, P. Tiwary, and M. Parrinello, ``Variationally optimized free energy flooding for rate calculation," {\it Phys. Rev. Lett.}, 115, 070601 (2015). \\
\\
D. Ozog, J. McCarty, G. Gossett, A. Malony, and M. Guenza, ``Fast equilibration of coarse-grained polymer liquids," {\it J. Comput. Sci.}, 9, 33 (2015).\\
\\
J. McCarty, A. J. Clark, J. Copperman, and M. G. Guenza, ``An analytical coarse- graining method which preserves the free energy, structural correlations, and thermodynamic state of polymer melts from the atomistic to the mesoscale," {\it J. Chem. Phys.}, 140, 204913 (2014). \\
\\
A. J. Clark, J. McCarty, and M. G. Guenza, ``Effective potentials for representing polymers in melts as chains of interacting soft particles," {\it J. Chem. Phys.}, 139, 124906 (2013). [co-first author] \\
\\
J. McCarty, A. J. Clark, I. Y. Lyubimov, and M. G. Guenza, ``Thermodynamic consistency between analytic integral equation theory and coarse-grained molecular dynamics simulations of homopolymer melts," {\it Macromol.}, 45, 8482 (2012). \\
\\
A. J. Clark, J. McCarty, I. Y. Lyubimov, and M. G. Guenza, ``Thermodynamic consistency in variable-level coarse-graining of polymeric liquids," {\it Phys. Rev. Lett.}, 109, 168301 (2012).\\
\\
J. McCarty and M. G. Guenza, ``Multiscale modeling of binary polymer mixtures: scale bridging in the athermal and thermal regime," {\it J. Chem. Phys.}, 133, 094904 (2010). \\
\\
I. Y. Lyubimov, J. McCarty, A. Clark, and M. G. Guenza ``Analytical rescaling of polymer dynamics from mesoscale simulations," {\it J. Chem. Phys.}, 132, 224903 (2010). \\
\\
J. McCarty, I. Y. Lyubimov, and M. G. Guenza. ``Effective soft-core potentials and mesoscopic simulations of binary polymer mixtures," {\it Macromol.}, 43, 3964-3979 (2010). \\
\\
J. McCarty, I. Y. Lyubimov, and M. G. Guenza. ``Multiscale modeling of coarse- grained macromolecular liquids," {\it J. Phys. Chem. B.}, 113, 11876-11886 (2009). 
%----------------------------------------------------------------------------------------

%	AWARDS
%----------------------------------------------------------------------------------------
 
\section{Awards} 
UCSB ScienceLine award in physics/chemistry/engineering for outstanding contribution to promoting science education in K-12 schools, University of California Santa Barbara, 2018 \\
\\
Dow Materials Institute and the Materials Research Laboratory travel award to attend APS national meeting in Los Angeles, CA, March 2018 \\
\\
University of Oregon Travel Award to attend ACS national meeting in Indianapolis, IN, September 2013 \\
\\
NSF Graduate STEM Fellow in K-12 Education (GK-12), University of Oregon, 2012 - 2013 \\
\\
Graduate Science Literacy Program Fellow, University of Oregon, 2012  \\
\\
University of Oregon Travel Award to attend APS national meeting in Boston, MA, March 2012 \\
\\
Fellowship to attend workshop on Multiscale Theory and Simulation at the University of Chicago, Chicago, IL, June 2012 \\
\\
MCC Travel Award to attend First Les Houches School in Computational Physics: Soft Matter, Les Houches School in Computational Physics, France, June 2011 \\
\\
The College of Arts and Sciences Henry V. Howe Scholarship Award, University of Oregon, 2010 \\
\\
Biochemistry student of the year, California Polytechnic State University, 2007

\section{Selected Talks}
``Conditions for complex coacervation of the microtubule-associated tau protein predicted from field theoretic simulations," American Physical Society, March Meeting, March 7, 2019, Boston, MA.\\
\\
``Application of fully-fluctuating field theoretic simulations to study liquid-liquid phase separation in biology," Complex Fluids Design Consortium Annual Meeting, January 29, 2019, Santa Barbara, CA.\\
\\
``Field-theoretic simulations of discrete Gaussian chain polyelectrolytes as a model for coacervation in intrinsically disordered peptides," American Physical Society, March Meeting, March 5, 2018, Los Angeles, CA.\\
\\
``Recent Applications of the Variationally Enhanced Sampling Method," MARVEL Junior Retreat, July 19, 2016, Les Diablerets, Switzerland. \\
\\
``Lessons from Analytical Coarse-graining: Representability, Thermodynamic Consistency, and Long Range Correlations," American Chemical Society National Meeting, September 9, 2013, Indianapolis, IN. \\
\\
``How Reliable Are Soft Potentials? Ensuring Thermodynamic Consistency Between Hierarchical Models of Polymer Melts," American Physical Society, March Meeting, March 1, 2012, Boston, MA. \\
\\
``Thermodynamic Consistency in Highly Coarse-Grained Models of Polymer Melts," American Physical Society, Northwest Section Meeting, Saturday, October 22, 2011, Corvallis, OR. \\
\\
``Effective Pair Potentials and Mesoscale Simulations of Binary Polymer Blends," American Physical Society, March Meeting, March 15, 2010, Portland, OR. \\
\\
``A Multiscale Modeling Procedure for Simulations of Polymer Melts," American Physical Society, Northwest Section Meeting, Friday, May 15, 2009, Vancouver, BC, Canada.

%----------------------------------------------------------------------------------------
%	TEACHING EXPERIENCE
%----------------------------------------------------------------------------------------
\section{Teaching Experience}
Co-Organizer and Instructor - MARVEL School on Variationally Enhanced Sampling, Universita della Svizzera italiana (USI), Lugano, Switzerland, February 2017 \\
\\
Teaching Assistant, Computational Chemistry, University of Oregon, Fall 2013 \\
\\
Teaching Assistant, General Chemistry, University of Oregon, Summer 2013 \\
\\
Science Literacy Program Fellow, University of Oregon 2012 \\
\begin{itemize}
\item Co-Instructor: Information, Quantum Mechanics, and DNA \\
For non-science majors, introduction to the physical and chemical principles governing how information is stored and transmitted through DNA. Duties include preparing and presenting lectures, in-class inquiry based activities, holding regular office hours, and assisting in course development. \\ 
Program website: http://scilit.uoregon.edu
\end{itemize} 
Teaching Assistant, General Chemistry Laboratory, University of Oregon, 2011-2012 \\
\\
Research Mentor for visiting undergraduate student, Crystal Valdez, through the  research experience for undergraduates (REU) program, University of Oregon, 6/09-8/09 \\
\\
Research Mentor for undergraduate student, Ha H. Truong, University of Oregon, 2010-2011 \\
\\
Teaching Assistant, Advanced General Chemistry Laboratory, University of Oregon, 2007-2008 \\

\section{Outreach}
ScienceLine Scientist, University of California Santa Barbara 2017-2018 \\
\begin{itemize}
\item Program that fosters the interest of K-12 students in science education. Students and teachers send science questions to ScienceLine and receive answers from scientists. 
\end{itemize}
Participant in JSU-UCSB Workshop on Gas Phase Spectroscopy and Theoretical Approaches, University of California Santa Barbara, 2018 \\
\begin{itemize}
\item Part of an NSF funded JSU-UCSB Partnership For Research and Education in Materials Science grant to foster research collaborations between UCSB and Jackson State University and to establish infrastructure for the education, training and mentoring of minority students
\end{itemize}
National Science Foundation GK-12 Program Fellow, University of Oregon 2012-2013 \\
\begin{itemize}
\item NSF funded program to assist classroom teachers in rural Oregon with science curriculum and inquiry-based learning. Fellowship duties included designing and teaching daily science lessons to K-8 students over a six week sabbatical in rural Oregon. \\
Additional information: http://stemcore.uoregon.edu/stem-programs/uo-programs/gk-12-fellows/
\end{itemize}

\end{document}